% \section{Introducción}
% En este capítulo se detalla el proceso de seguimiento y control llevado a cabo durante el desarrollo del proyecto, con el objetivo de garantizar el cumplimiento de los objetivos establecidos y gestionar posibles desviaciones respecto a la planificación inicial.

% \section{Metodología de Seguimiento}
% Para realizar el seguimiento del proyecto, se han empleado diversas herramientas y metodologías, incluyendo:
% \begin{itemize}
%     \item \textbf{Gestión de código}: Uso de \textit{GitHub} para el control de versiones y la integración continua.
%     \item \textbf{Reuniones con la tutora}: Seguimiento del progreso mediante reuniones periódicas para evaluar avances y recibir retroalimentación.
%     \item \textbf{Registro de cambios}: Documentación de modificaciones en funcionalidades y planificación.
% \end{itemize}

% \section{Planificación vs. Ejecución}
% A lo largo del desarrollo del proyecto, han surgido algunas diferencias entre la planificación inicial y la ejecución real. En la siguiente tabla se presenta un resumen de estas desviaciones:

% \begin{table}[H]
%     \centering
%     \begin{tabular}{|p{4cm}|p{5cm}|p{5cm}|}
%         \hline
%         \textbf{Tarea}              & \textbf{Planificación Inicial} & \textbf{Ejecución Real}                                         \\
%         \hline
%         Desarrollo de autenticación & 2 semanas                      & 3 semanas (por problemas con la API de Spotify)                 \\
%         \hline
%         Implementación de gráficos  & 4 semanas                      & 5 semanas (se añadió \textit{D3.js} para mayor personalización) \\
%         \hline
%         Pruebas de carga            & No contemplado                 & 2 semanas (se decidió incluir K6 para evaluar rendimiento)      \\
%         \hline
%     \end{tabular}
%     \caption{Comparación entre planificación y ejecución}
%     \label{tab:planificacion}
% \end{table}

% \section{Gestión de Cambios}
% Durante el desarrollo, se han realizado ajustes en la planificación y en algunas funcionalidades debido a limitaciones técnicas y mejoras identificadas. Algunos de los cambios más relevantes incluyen:
% \begin{itemize}
%     \item Modificación en la autenticación para manejar correctamente los \textit{access tokens} de Spotify.
%     \item Inclusión de \textit{D3.js} para mejorar la personalización de gráficos.
%     \item Implementación de pruebas de carga con \textit{K6} para evaluar el rendimiento de la aplicación.
% \end{itemize}

% \section{Indicadores de Seguimiento}
% Para medir el avance del proyecto, se han utilizado los siguientes indicadores:
% \begin{itemize}
%     \item \textbf{Porcentaje de tareas completadas}: Relación entre tareas finalizadas y el total planificado.
%     \item \textbf{Commits en GitHub}: Frecuencia de actualizaciones en el código fuente.
%     \item \textbf{Feedback de la tutora}: Evaluación cualitativa del progreso y calidad del trabajo realizado.
% \end{itemize}

% \section{Evaluación Final}
% En términos generales, el proyecto ha seguido una evolución acorde a la planificación inicial, con algunas desviaciones necesarias para mejorar su calidad y rendimiento. A pesar de ciertos retrasos en la ejecución, se han cumplido los objetivos principales del TFG. Como mejora futura, se podría considerar una planificación más flexible para adaptarse mejor a cambios imprevistos.

