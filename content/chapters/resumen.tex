Este Trabajo de Fin de Grado presenta el desarrollo de una plataforma web para el análisis interactivo de datos musicales de \textit{Spotify}. Se ha realizado un estudio profundo de la Web API de dicha plataforma, identificando las limitaciones en la obtención de datos y las particularidades necesarias para asegurar la correcta protección de la información del usuario. A partir de estas consideraciones, se ha diseñado e implementado una aplicación que, tras la autenticación del usuario mediante OAuth 2.0, recupera y procesa datos sobre sus hábitos de escucha, ofreciendo métricas y visualizaciones avanzadas.

La solución se basa en \textbf{Next.js 15} (con \textit{App Router} y \textit{Route Handlers}) para combinar en un solo entorno la parte de frontend y la lógica de backend. En el frontend, se ha usado \textbf{React 19} y \textbf{Tailwind CSS} para crear un diseño modular, intuitivo y adaptable. En el backend, las rutas de la aplicación se han implementado a través de \textit{handlers} seguros que procesan las peticiones y actúan como intermediarios con la API de \textit{Spotify}, respetando sus políticas de uso y de protección de datos. Como parte esencial de la plataforma, se ofrecen diferentes estadísticas, desde las más básicas (canciones o artistas más escuchados) hasta otras más complejas y originales (comparación de patrones de escucha con la popularidad media, presentación de canciones favoritas en formatos creativos, entre otras).

La aplicación se ha sometido a un ciclo de pruebas unitarias y de integración, apoyado en la automatización de flujos de integración y despliegue continuo (CI/CD) mediante \textbf{GitHub Actions} y la plataforma de hosting \textbf{Vercel}. Gracias a ello, se logra una actualización rápida y confiable de la aplicación. El resultado global es un sistema que permite a los usuarios descubrir tendencias y analizar en detalle su historial musical, aprovechando los datos y la flexibilidad que proporciona la API de \textit{Spotify}.