\section{Alcance}

Definir con precisión el alcance es fundamental para asegurar que el desarrollo se ajuste a los objetivos propuestos y se realice dentro de los recursos y tiempos establecidos. Para ello, en esta sección se delimitarán las funcionalidades, exclusiones y limitaciones que se esperan en este proyecto.

\subsection{Funcionalidades Incluidas}

En la plataforma web se ofrecen las siguientes funcionalidades principales:

\begin{itemize}
    \item \underline{Autenticación segura} mediante las credenciales de \textit{Spotify}.
    \item \underline{\textit{Home} o Panel Inicial} donde se muestran la información básica de la cuenta.
    \item Análisis detallado y visualizaciones \underline{gráficas avanzadas, interactivas y actualizadas} de sus datos musicales.
    \item Interfaz \underline{adaptativa, intuitiva y responsiva}.
    \item \underline{Cierre de sesión seguro}.
\end{itemize}

\subsection{Exclusiones}

Para establecer expectativas claras sobre el alcance del proyecto, se detallan a continuación las funcionalidades que \textbf{no} serán incluidas en la plataforma web:

\begin{itemize}
    \item No se desarrollarán \underline{aplicaciones nativas de otras plataformas} como móvil, PC, Mac o Linux; el acceso será exclusivamente a través de la web.

    \item Aunque se seguirá un diseño intuitivo, no se implementarán funcionalidades específicas de \underline{accesibilidad avanzadas} como compatibilidad con lectores de pantalla o navegación por teclado.

    \item La plataforma se enfoca exclusivamente en la integración con \textit{Spotify}; se excluyen todos los \underline{otros servicios de streaming} como \textit{Apple Music}, \textit{Deezer}, etc.

    \item No se \underline{almacenarán de forma persistente datos personales} del usuario en servidores propios más allá de lo necesario para la sesión actual; todos los datos se obtendrán directamente de la API de \textit{Spotify} y se manejarán en tiempo real.

    \item Se excluye el desarrollo de funcionalidades relacionadas con la \underline{interacción social} (envío de mensajes, compartir estadísticas, rankings entre usuarios, etc.) dentro o a través de la plataforma, ya que superarían el alcance recogido dentro de un TFG.

          % * COMPROBAR SI SE HAN IMPLEMENTADO O NO ESTAS FUNCIONALIDADES * %

          % * \item \textbf{Integración con Redes Sociales}: No se implementarán opciones para compartir estadísticas o visualizaciones directamente en redes sociales como Facebook, Twitter o Instagram.

          % * \item \textbf{Exportación de Datos o Estadísticas}: No se incluirá la opción de exportar los datos o estadísticas a formatos externos como PDF, CSV o imágenes descargables.

          % * \item \textbf{Soporte Multilenguaje}: La interfaz de usuario estará disponible únicamente en español y no se ofrecerá soporte para otros idiomas.

\end{itemize}

\subsection{Limitaciones}

Durante el desarrollo del proyecto, se han identificado las siguientes limitaciones que han podido afectar al alcance y a las funcionalidades de la web:

\begin{itemize}
    \item Las políticas de seguridad de \textit{Spotify} impiden el \underline{almacenamiento persistente de datos} personales, limitando funcionalidades que requieran conservar información del usuario entre sesiones.

    \item El procesamiento de los datos se ve limitada por los \underline{recursos computacionales} que la nube de \textit{Vercel} ofrece, descartando técnicas avanzadas como el aprendizaje automático.

    \item El \underline{tiempo y los recursos disponibles} para el desarrollo del proyecto son finitos, lo que ha obligado a priorizar funcionalidades esenciales y descartar características adicionales.

    \item Al hacer uso de una API de terceros, todas las funcionalidades necesitan una \underline{conexión} \underline{activa a Internet} para poder funcionar.
\end{itemize}

\section{Gestión de Tareas}
\subsection{Descripción de Tareas}
\subsection{Dedicaciones}
\subsection{Dependencias entre Tareas}
\subsection{Periodos de Desarrollo}
\subsection{Hitos}

% TODO -> No está terminado, hay que revisar todos los riesgos y ponerlo bien
\section{Gestión de Riesgos}

A continuación, se detallan los riesgos identificados que pueden afectar el desarrollo y el alcance del proyecto:

\begin{itemize}
    \item Dependencia de la API de \textit{Spotify}: La API de \textit{Spotify} podría cambiar, tener interrupciones o limitar el acceso a ciertos datos.

    \item Limitaciones de la API de \textit{Spotify}: Restricciones en los tipos de datos accesibles, datos históricos limitados y permisos que los usuarios pueden no conceder.

    \item Tasa de peticiones (\textit{rate limiting}) de la API de \textit{Spotify}: Exceder el número máximo de peticiones permitidas en un período de tiempo puede afectar la capacidad de actualización de datos en tiempo real.

    \item Falta de almacenamiento persistente de datos del usuario: La imposibilidad de guardar datos entre sesiones limita ciertas funcionalidades como la personalización o el historial de preferencias del usuario.

    \item Dependencia de servicios de terceros (\textit{Vercel}): Posibles interrupciones en los servicios de despliegue continuo o cambios en las políticas de \textit{Vercel}.

    \item Tiempo limitado para el desarrollo: El tiempo y los recursos disponibles son finitos, lo que obliga a priorizar ciertas funcionalidades esenciales y descartar características adicionales.

    \item Necesidad de conexión a Internet: La plataforma requiere una conexión a Internet activa para funcionar correctamente, lo que puede afectar a los usuarios en entornos con conectividad limitada.

    \item Compatibilidad con navegadores y dispositivos antiguos: La plataforma puede no ser completamente funcional en navegadores o dispositivos más antiguos, lo que afecta a un porcentaje pequeño de usuarios.

    \item Falta de experiencia con las tecnologías utilizadas: La curva de aprendizaje de tecnologías nuevas como \textit{Next.js} y \textit{TypeScript} podría ralentizar el desarrollo.

    \item Rendimiento en dispositivos móviles o de gama baja: Las visualizaciones avanzadas pueden no funcionar de manera óptima en dispositivos con menor capacidad de procesamiento.

    \item Seguridad y privacidad de los datos del usuario: El manejo incorrecto de los datos de los usuarios podría infringir las políticas de \textit{Spotify} o el RGPD, con consecuencias legales o de suspensión del servicio.

\end{itemize}


% TODO -> Solo he pegado esto, pero hay que mirarlo todo
\section{Gestión de Calidad}

El objetivo de la gestión de calidad es garantizar que el proyecto cumpla con los requisitos funcionales y no funcionales, asegurando un producto robusto, eficiente y alineado con las expectativas de los usuarios.

\subsection{Línea Base}
Los estándares mínimos establecidos para asegurar la calidad del proyecto son:
\begin{itemize}
    \item Buenas prácticas de desarrollo, aseguradas mediante revisiones de código.
    \item Uso de metodologías ágiles para una planificación iterativa y control de calidad.
    \item Diseño de interfaz de usuario intuitiva y conforme a los estándares de accesibilidad.
\end{itemize}

\subsection{Criterios de Éxito y Aceptación}
Para asegurar que el producto cumple con las expectativas, los criterios de aceptación incluyen:
\begin{itemize}
    \item Cumplimiento de los requisitos funcionales y no funcionales.
    \item Ausencia de errores críticos que afecten la experiencia de usuario.
    \item Cumplimiento de los objetivos de rendimiento (tiempos de carga, capacidad de respuesta).
    \item Feedback positivo en las pruebas de usabilidad realizadas con usuarios finales.
\end{itemize}

\subsection{Plan de Calidad}
El plan de calidad incluye:
\begin{itemize}
    \item Pruebas automatizadas, incluyendo pruebas unitarias, de integración y de aceptación.
    \item Revisión de código por pares para asegurar la consistencia y calidad del código.
    \item Implementación de CI/CD usando Vercel para garantizar despliegues automáticos y controlados.
    \item Pruebas de usabilidad con usuarios reales para validar la experiencia del usuario.
\end{itemize}

\subsection{Herramientas y Tecnologías}
Para asegurar la calidad del proyecto, se utilizarán las siguientes herramientas:
\begin{itemize}
    \item \textbf{CI/CD}: Vercel y GitHub Actions para control de calidad durante el despliegue.
    \item \textbf{Testing}: Jest para pruebas unitarias y Cypress para pruebas de integración.
    \item \textbf{Control de calidad del código}: ESLint y Prettier para mantener estándares de código.
    \item \textbf{Monitoreo del rendimiento}: Lighthouse para verificar tiempos de carga y rendimiento.
\end{itemize}

\subsection{Indicadores de Calidad}
Los KPIs definidos para evaluar la calidad del proyecto son:
\begin{itemize}
    \item Número de bugs críticos detectados.
    \item Tiempos de carga y respuesta del sistema.
    \item Porcentaje de cobertura de pruebas.
    \item Satisfacción del usuario en pruebas de usabilidad.
\end{itemize}


\section{Tecnologías y Herramientas Utilizadas}
