\section{Contexto} \label{ch:contexto}

El uso de datos personalizados es un componente esencial en el desarrollo de aplicaciones y servicios digitales actuales. Las plataformas buscan ofrecer experiencias ajustadas a las preferencias de los usuarios, utilizando grandes volúmenes de datos para generar contenido adaptado. En este contexto, \textit{Spotify} se ha consolidado como una de las plataformas más destacadas de \textit{streaming} musical, no solo por su extenso catálogo, sino también por su capacidad de ofrecer recomendaciones y estadísticas de uso personalizadas.

Una de las características más valoradas por los usuarios de \textit{Spotify} es la posibilidad de acceder a estadísticas personales que les permiten comprender mejor sus hábitos de escucha. En 2016, como producto de una campaña de márketing viral, \textit{Spotify} lanzó \textit{Spotify Wrapped}, un resumen anual de cada usuario, que generó un gran interés y participación en las redes sociales. Sin embargo, el acceso a estas estadísticas solo está disponible en el mes de diciembre, lo que genera una oportunidad para desarrollar herramientas complementarias que ofrezcan este tipo de análisis de manera más accesible y frecuente.

El acceso a estos datos es posible gracias a la \textit{Web API} pública que \textit{Spotify} pone a disposición de los desarrolladores. De esta manera, es posible obtener una amplia gama de datos sobre el comportamiento del usuario, como sus canciones más escuchadas, artistas favoritos y playlists. Además, la API ofrece acceso a muchos datos adicionales que no son utilizados en \textit{Spotify Wrapped}, pero realizando diferentes combinaciones, se pueden crear nuevas formas enriquecidas de análisis que presenten la información de manera más profunda y personalizada.

También es necesario mencionar, que el crecimiento de las aplicaciones web interactivas ha sido posible gracias a tecnologías y frameworks modernos, que permiten crear interfaces rápidas y eficientes, optimizando el
rendimiento y la experiencia de usuario y simplificando el desarrollo. Además, los sistemas de integración y despliegue continuo (CI/CD) facilitan la entrega de aplicaciones de manera automatizada y escalable, garantizando que estén siempre actualizadas y accesibles para los usuarios.

\section{Motivación}

La elección de este Trabajo de Fin de Grado surge de un interés personal que he tenido desde el inicio de mi carrera universitaria. Desde el primer año, he tenido la idea base para implementar un servicio en torno a la \textit{Web API} de \textit{Spotify}. Sin embargo, en ese momento no contaba con los conocimientos necesarios para llevarlo a cabo. Ahora, tras completar la carrera, me siento con las capacidades necesarias para poder realizar dicho proyecto y materializar esta idea.

Como usuario habitual de \textit{Spotify}, siempre me ha fascinado la función de \textit{Spotify Wrapped}. No obstante, me resulta limitante que esta información solo esté disponible durante un breve periodo del año. Existen otras herramientas y páginas web que analizan datos de \textit{Spotify}, pero suelen ofrecer estadísticas genéricas y demasiado básicas que carecen de profundidad. Estoy convencido de que es posible obtener estadísticas mucho más interesantes y, conversando con compañeros y otros usuarios, existe un interés general por acceder a estas estadísticas en cualquier momento, no solo una vez al año. La música que cada individuo escucha es algo personal y, a menudo, forma parte de su identidad. Por ello, considero que este proyecto no solo es interesante y motivador para mí, sino que también puede aportar valor a otros usuarios que comparten esta misma idea.

La posibilidad de crear un proyecto que me interese de principio a fin, y que yo mismo desearía utilizar, es una gran motivación. Además, la selección de tecnologías que he decidido emplear, como se detallará más adelante, no solo son ampliamente demandadas en el mercado actual, sino que también contribuyen a mi crecimiento profesional y enriquecen mis conocimientos.

\section{Objetivos del Proyecto}

Para empezar a caracterizar el proyecto de manera concreta, en este apartado se definen los objetivos que guiarán el desarrollo. El \textbf{objetivo general} de este proyecto es desarrollar una plataforma web interactiva que permita a los usuarios de \textit{Spotify} acceder a las visualizaciones y análisis de sus datos musicales personales cuando lo deseen. Para alcanzar este objetivo general, se plantean los siguientes \textbf{objetivos específicos}:

\begin{itemize}
    \item Analizar la API de \textit{Spotify} para obtener los datos necesarios.
    \item Desarrollar la lógica necesaria para filtrar, organizar y transformar los datos obtenidos, preparándolos para su representación gráfica y asegurando que las estadísticas sean relevantes y significativas.
    \item Diseñar una interfaz de usuario intuitiva, interactiva y adaptable a diferentes dispositivos.
    \item Adoptar tecnologías modernas como \textit{Next.js} y \textit{TypeScript} para beneficiarse de las ventajas que ofrecen.
    \item Implementar prácticas de CI/CD usando la plataforma de \textit{Vercel}.
    \item Implementar políticas de seguridad con respecto a los datos de usuarios establecidas por \textit{Spotify} para desarrolladores.
    \item Documentar el proceso de desarrollo.
\end{itemize}

\section{Estructura de la Memoria}

La estructura de esta memoria refleja las diferentes etapas por las que atraviesa un proyecto de desarrollo software. Cada capítulo aborda aspectos clave que contribuyen al logro de los objetivos propuestos.

En la \nameref{ch:introduccion} se ha establecido el contexto del proyecto, la motivación que impulsa su realización y los objetivos tanto del proyecto como los relacionados con los ODS. Para terminar de definir el contexto en el que se está desarrollando, en el \nameref{ch:contextoCompetitivo} se realizará un análisis de las soluciones que existentes. A continuación, en la \nameref{ch:planificacion}, se define el alcance del proyecto y se aborda lodo lo relacionado con la gestión de tareas, riesgos y calidad del proyecto. Además, se presentan las tecologías que se han utilizado durante todo el desarrollo.

El \nameref{ch:analisis} se centra en el estudio de la API de \textit{Spotify}, el cual es \textbf{fundamental para el desarrollo del proyecto}. Se especifican los requisitos y se presentan los casos de uso que guiarán el desarrollo de la aplicación. En conjunto con el capítulo anterior, en el \nameref{ch:diseño} se describe la arquitectura del sistema y la estructura de la interfaz de usuario (UI), tanto a nivel técnico como visual. Además, se abordan aspectos de seguridad y se planifica el diseño de las pruebas.

La \nameref{ch:implementacion} detalla el proceso de desarrollo de la aplicación, incluyendo las decisiones tomadas durante esta fase. Se describen los retos enfrentados y cómo se han resuelto. Seguido, en el capítulo de las \nameref{ch:pruebas} se exponen las pruebas realizadas y se analizan los resultados obtenidos. Como consecuencia lógica, en \nameref{ch:despliegue} se explica el proceso de puesta en producción de la aplicación y las estrategias de CI/CD implementadas.

Finalmente, en \nameref{ch:conclusiones} se hace una reflexión sobre los resultados alcanzados, evaluando el cumplimiento de los objetivos iniciales, el seguimiento en comparación a la planificación inicial y se discuten las lecciones aprendidas y se proponen líneas futuras de trabajo.

Mediante este flujo, se espera que el público lector no tenga problemas para seguir la línea de pensamiento que se ha llevado a cabo y que cada apartado quede aclarado o, en menor medida, justificado por los capítulos anteriores.
