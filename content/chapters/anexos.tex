\chapter{Capturas de la Interfaz de Usuario} \label{ch:anexoA}

\section{Interfaz de Estadísticas}

Cada una de las estadísticas ha sido diseñada de manera independiente y autocontenida. Es por esto por lo que, aunque se sigan las directrices de diseño generales de la página, cada una tiene un caracter y presentación diferenciable. También cabe destacar que, en algunas de ellas, se ha preferido desarrollar la implementación funcional de manera más robusta, antes de invertir el tiempo en mejorar por completo la estética de la presentación.

\begin{figure}[H]
    \centering
    \includegraphics[width=0.45\textwidth]{figures/capturas_ui/hall_of_fame.png}
    \caption{Interfaz de la estadística \textit{Hall Of Fame}.}
    \label{fig:hall_of_fame}
\end{figure}

\begin{figure}[H]
    \centering
    \vspace{-10pt}
    \begin{minipage}{0.32\textwidth}
        \centering
        \includegraphics[width=\textwidth]{figures/capturas_ui/la_bitacora_año.png}
        \caption{Interfaz de la estadística \textit{La Bitácora} (año).}
        \label{fig:la_bitacora_año}
    \end{minipage}
    \begin{minipage}{0.32\textwidth}
        \centering
        \includegraphics[width=\textwidth]{figures/capturas_ui/la_bitacora_mes.png}
        \caption{Interfaz de la estadística \textit{La Bitácora} (mes).}
        \label{fig:la_bitacora_mes}
    \end{minipage}
    \begin{minipage}{0.32\textwidth}
        \centering
        \includegraphics[width=\textwidth]{figures/capturas_ui/la_bitacora_dia.png}
        \caption{Interfaz de la estadística \textit{La Bitácora} (día).}
        \label{fig:la_bitacora_dia}
    \end{minipage}
    \caption{Interfaz de la estadística \textit{La Bitácora}, en los tres diferentes niveles de detalle.}
    \label{fig:la_bitacora}
\end{figure}

\begin{figure}[H]
    \centering
    \vspace{-10pt}
    \includegraphics[width=0.55\textwidth]{figures/capturas_ui/huella_del_dia.png}
    \caption{Interfaz de la estadística \textit{Huella Del Día}.}
    \label{fig:huella_del_dia}
\end{figure}

\begin{figure}[H]
    \centering
    \vspace{-10pt}
    \begin{minipage}{0.47\textwidth}
        \centering
        \includegraphics[width=\textwidth]{figures/capturas_ui/tus_decadas_out.png}
        \caption{Interfaz de la estadística \textit{Tus Décadas} (zoom out).}
        \label{fig:tus_decadas_out}
    \end{minipage}
    \begin{minipage}{0.47\textwidth}
        \centering
        \includegraphics[width=\textwidth]{figures/capturas_ui/tus_decadas_zoom.png}
        \caption{Interfaz de la estadística \textit{Tus Décadas} (zoom in).}
        \label{fig:tus_decadas_zoom}
    \end{minipage}
    \caption{Interfaz de la estadística \textit{Tus Décadas}, en diferentes niveles de zoom.}
    \label{fig:tus_decadas}
\end{figure}

\begin{figure}[H]
    \centering
    \vspace{-10pt}
    \begin{minipage}{0.47\textwidth}
        \centering
        \includegraphics[width=\textwidth]{figures/capturas_ui/estaciones_musicales_cerrado.png}
        \caption{Interfaz de la estadística \textit{Estaciones Musicales} (cerrado).}
        \label{fig:estaciones_musicales_cerrado}
    \end{minipage}
    \begin{minipage}{0.47\textwidth}
        \centering
        \includegraphics[width=\textwidth]{figures/capturas_ui/estaciones_musicales_abierto.png}
        \caption{Interfaz de la estadística \textit{Estaciones Musicales} (abierto).}
        \label{fig:estaciones_musicales_abierto}
    \end{minipage}
    \caption{Interfaz de la estadística \textit{Estaciones Musicales}, en los estados de cerrado y abierto.}
    \label{fig:estaciones_musicales}
\end{figure}

\begin{figure}[H]
    \centering
    \vspace{-10pt}
    \begin{minipage}{0.47\textwidth}
        \centering
        \includegraphics[width=\textwidth]{figures/capturas_ui/indice_de_interferencia_originales.png}
        \caption{Interfaz de la estadística \textit{Índice De Interferencia} (originales).}
        \label{fig:indice_de_interferencia_originales}
    \end{minipage}
    \begin{minipage}{0.47\textwidth}
        \centering
        \includegraphics[width=\textwidth]{figures/capturas_ui/indice_de_interferencia_combinado.png}
        \caption{Interfaz de la estadística \textit{Índice De Interferencia} (combinados).}
        \label{fig:indice_de_interferencia_combinado}
    \end{minipage}
    \caption{Interfaz de la estadística \textit{Índice De Interferencia}, con las ondas originales y su forma combinada.}
    \label{fig:indice_de_interferencia}
\end{figure}

\section{Componentes Secundarios, de Carga y de Errores}

Además de los elementos presentados, también se han implementado algunos componentes secundarios para aportar más interactividad a la web y mejorar la experiencia al usuario.

\begin{figure}[H]
    \centering
    \includegraphics[width=0.75\textwidth]{figures/capturas_ui/tops_detalle.png}
    \caption{Detalle de los tres \textit{tops} junto con el selector de periodo de tiempo.}
    \label{fig:tops_detalle}
\end{figure}

\begin{figure}[H]
    \centering
    \vspace{-10pt}
    \includegraphics[width=0.75\textwidth]{figures/capturas_ui/recently_played.png}
    \caption{Detalle de \textit{Recently Played} con el botón para ampliar o contraer la lista.}
    \label{fig:recently_played}
\end{figure}

\begin{figure}[H]
    \centering
    \vspace{-10pt}
    \includegraphics[width=0.4\textwidth]{figures/capturas_ui/user_action_panel.png}
    \caption{Panel del usuario con la opción de \textit{Log Out}.}
    \label{fig:user_action_panel}
\end{figure}

\begin{figure}[H]
    \centering
    \vspace{-10pt}
    \includegraphics[width=0.75\textwidth]{figures/capturas_ui/pantalla_carga.png}
    \caption{Componente que se muestra mientras se cargan los datos de las estadísticas, con texto dinámico.}
    \label{fig:pantalla_carga}
\end{figure}

\begin{figure}[H]
    \centering
    \vspace{-10pt}
    \includegraphics[width=0.75\textwidth]{figures/capturas_ui/error_no_favoritos.png}
    \caption{Componente de error que se muestra cuando el usuario no tiene ninguna canción guardada en su lista de favoritos.}
    \label{fig:error_no_favoritos}
\end{figure}

\begin{figure}[H]
    \centering
    \vspace{-10pt}
    \includegraphics[width=0.8\textwidth]{figures/capturas_ui/skeletons.png}
    \caption{Componentes de \textit{loading} para los tres \textit{tops}.}
    \label{fig:skeletons}
\end{figure}

\chapter{Diagramas de Secuencia Adicionales} \label{ch:anexoB}

\begin{itemize}
    \item \textbf{Cerrar Sesión:} figura \ref{fig:ds_cerrar_sesion}
    \item \textbf{Ver Huella Del Día:} figura \ref{fig:ds_ver_huella_del_dia}
    \item \textbf{Ver Estaciones Musicales:} figura \ref{fig:ds_ver_estaciones_musicales}
    \item \textbf{Ver Tus Décadas:} figura \ref{fig:ds_ver_tus_decadas}
    \item \textbf{Ver Índice de Interferencia:} figura \ref{fig:ds_ver_indice_de_interferencia}
\end{itemize}

\begin{figure}[H]
    \centering
    \includegraphics[width=0.85\textwidth]{figures/diagramas_secuencia/ds_cerrar_sesion.png}
    \caption{Diagrama de secuencia: \textbf{Cerrar Sesión}.}
    \label{fig:ds_cerrar_sesion}
\end{figure}

\begin{figure}[H]
    \centering
    \includegraphics[width=\textwidth]{figures/diagramas_secuencia/ds_ver_huella_del_dia.png}
    \caption{Diagrama de secuencia: \textbf{Ver Huella Del Día}.}
    \label{fig:ds_ver_huella_del_dia}
\end{figure}

\begin{figure}[H]
    \centering
    \includegraphics[width=\textwidth]{figures/diagramas_secuencia/ds_ver_estaciones_musicales.png}
    \caption{Diagrama de secuencia: \textbf{Ver Estaciones Musicales}.}
    \label{fig:ds_ver_estaciones_musicales}
\end{figure}

\begin{figure}[H]
    \centering
    \includegraphics[width=\textwidth]{figures/diagramas_secuencia/ds_ver_tus_decadas.png}
    \caption{Diagrama de secuencia: \textbf{Ver Tus Décadas}.}
    \label{fig:ds_ver_tus_decadas}
\end{figure}

\begin{figure}[H]
    \centering
    \includegraphics[width=\textwidth]{figures/diagramas_secuencia/ds_ver_indice_de_interferencia.png}
    \caption{Diagrama de secuencia: \textbf{Ver Índice De Interferencia}.}
    \label{fig:ds_ver_indice_de_interferencia}
\end{figure}

\chapter{Contenidos Relacionados con la Implementación} \label{ch:anexoC}

\section*{Fichero .env.local}

\begin{ifalgorithm}[H]
    \begin{lstlisting}[language=bash]
    # ID del cliente registrado en la API de Spotify
    SPOTIFY_CLIENT_ID="81af...642b"

    # Secreto del cliente registrado en la API de Spotify no compartir nunca
    SPOTIFY_CLIENT_SECRET="1207...393e"

    # URL del dominio donde se ejecuta la aplicacion en desarrollo, localhost
    DOMAIN_URL="http://localhost:3000"

    # URI de redireccion configurada en Spotify para la autenticacion OAuth
    SPOTIFY_REDIRECT_URI="http://localhost:3000/api/auth/callback"

    # URL utilizada por NextAuth para gestionar la autenticacion en la aplicacion
    NEXTAUTH_URL="http://localhost:3000/api/auth/callback"
    \end{lstlisting}
    \caption{Variables de entrono necesarios en el fichero \texttt{.env.local}.}
    \label{alg:variables_entorno}
\end{ifalgorithm}

\section*{Figura del Panel de Monitoreo de Spotify}

\begin{figure}[H]
    \centering
    \includegraphics[width=0.75\textwidth]{figures/registro_spotify/dashboard_spotify.png}
    \caption{Panel de monitoreo de la actividad de la aplicación registrada.}
    \label{fig:dashboard_spotify}
\end{figure}

\section*{Definición del Hook Personalizado useFetch}

\begin{ifalgorithm}[H]
    \begin{lstlisting}
    import { useEffect, useState } from "react";

    interface UseFetchResult<T> {
      data: T | null;
      loading: boolean;
      error: string | null;
    }

    export function useFetch<T>(url: string): UseFetchResult<T> {
      const [data, setData] = useState<T | null>(null);
      const [loading, setLoading] = useState(true);
      const [error, setError] = useState<string | null>(null);

      useEffect(() => {
        const controller = new AbortController();
        const signal = controller.signal;
        let isAborted = false;

        console.log("=== INICIO DE FETCH, setLoading(true) ===");
        setLoading(true);
        setData(null);
        setError(null);

        fetch(url, { credentials: "include", signal })
          .then((response) => {
            if (!response.ok) throw new Error("Failed to fetch data");
            return response.json() as Promise<T>;
          })
          .then((result) => {
            if (!isAborted) setData(result);
          })
          .catch((err) => {
            if (err.name !== "AbortError") {
              console.error("Error fetching data:", err);
              if (!isAborted) setError("Error al cargar los datos. Por favor, intentalo de nuevo mas tarde.");
            } else {
              console.log("=== FETCH ABORTADO ===");
            }
          })
          .finally(() => {
            if (!isAborted) {
              console.log("=== FIN DE FETCH, setLoading(false) ===");
              setLoading(false);
            }
          });

        return () => {
          console.log("=== ABORTANDO FETCH ===");
          isAborted = true;
          controller.abort();
        };
      }, [url]);

      return { data, loading, error };
    }
    \end{lstlisting}
    \caption{Definición del \textit{hook} personalizado \texttt{useFetch} para la obtención de datos de la API con gestión de estado y abortos de petición.}
    \label{alg:use_fetch}
\end{ifalgorithm}

\section*{Código de los Componentes Home y Stats}

\begin{ifalgorithm}[H]
    \begin{lstlisting}
    export default async function Home({
      searchParams,
    }: {
      searchParams: Promise<Record<string, string | string[] | undefined>>;
    }) {
      const resolvedSearchParams = await searchParams;

      const timeRange = Array.isArray(resolvedSearchParams.time_range)
        ? resolvedSearchParams.time_range[0]
        : resolvedSearchParams.time_range || "short_term";

      return (
        <main className='min-h-screen relative'>
          <section className='bg-spotify-black'>
            <div className='max-w-5xl mx-auto px-4 md:px-8 py-8'>
              <h1 className='text-4xl md:text-5xl font-bold mb-12 text-center bg-gradient-to-r from-spotify-green to-spotify-blue bg-clip-text text-transparent'>
                Tus Spotify Tops
              </h1>
              <Suspense fallback={<UserProfileSkeleton />}>
                <UserProfile />
              </Suspense>
              <div className='flex justify-center mt-8'>
                <TimeRangeSelector />
              </div>
              <div className='grid grid-cols-1 md:grid-cols-2 gap-4'>
                <Suspense fallback={<TrackArtistSkeleton count={5} />}>
                  <TopTracks timeRange={timeRange} />
                </Suspense>
                <Suspense fallback={<TrackArtistSkeleton count={5} />}>
                  <TopArtists timeRange={timeRange} />
                </Suspense>
                <div className='md:col-span-2'>
                  <Suspense fallback={<GenreSkeleton count={5} />}>
                    <TopGenres timeRange={timeRange} />
                  </Suspense>
                </div>
              </div>
            </div>
          </section>
          <section className='bg-[#0A0A0A]'>
            <div className='max-w-5xl mx-auto px-4 md:px-8 py-12'>
              <RecentlyPlayed />
            </div>
          </section>
        </main>
      );
    }
    \end{lstlisting}
    \caption{Definición del componente \texttt{Home}, encargado de renderizar la página principal con las estadísticas de usuario en Spotify.}
    \label{alg:home_component}
\end{ifalgorithm}

\begin{ifalgorithm}[H]
    \begin{lstlisting}
    "use client";

    import { useState } from "react";
    import StatsGrid from "@/components/StatsGrid";
    import StatWrapper from "@/components/StatWrapper";

    // Define the type for a single stat item
    type StatItem = {
      id: string;
      title: string;
      iconName: keyof typeof import("lucide-react");
      className?: string;
    };

    // Define the stats array with the correct types
    const stats: StatItem[] = [
      {
        id: "hall-of-fame",
        title: "Hall of Fame",
        iconName: "Award",
        className: "md:col-span-2 lg:col-span-2 lg:row-span-2",
      },
      { id: "tus-decadas", title: "Tus Decadas", iconName: "Rewind", className: "md:col-span-1 lg:col-span-2" },
      {
        id: "huella-del-dia",
        title: "Huella Del Dia",
        iconName: "Fingerprint",
      },
      { id: "estaciones-musicales", title: "Estaciones Musicales", iconName: "SunSnow" },

      {
        id: "la-bitacora",
        title: "La Bitacora",
        iconName: "BookMarked",
        className: "md:col-span-2 lg:col-span-2",
      },
      { id: "indice-de-interferencia", title: "indice de Interferencia", iconName: "AudioWaveform" },
    ];

    export default function Stats() {
      const [activeStat, setActiveStat] = useState<string | null>(null);
      const [isModalOpen, setIsModalOpen] = useState(false);

      const handleStatClick = (statId: string) => {
        setActiveStat(statId);
        setIsModalOpen(true);
      };

      const handleCloseModal = () => {
        setActiveStat(null);
        setIsModalOpen(false);
      };

      return (
        <main className='bg-spotify-black min-h-screen'>
          <div className='max-w-7xl mx-auto px-4 sm:px-6 lg:px-8 py-12'>
            <h1 className='text-4xl md:text-5xl font-bold mb-12 text-center bg-gradient-to-r from-spotify-green to-spotify-blue bg-clip-text text-transparent'>
              Estadisticas Avanzadas
            </h1>
            <StatsGrid stats={stats} onStatClick={handleStatClick} />
          </div>
          <StatWrapper activeStat={activeStat} isOpen={isModalOpen} onClose={handleCloseModal} />
        </main>
      );
    }
    \end{lstlisting}
    \caption{Definición del componente \texttt{Stats}, encargado de gestionar y renderizar la interfaz de las estadísticas avanzadas.}
    \label{alg:stats_component}
\end{ifalgorithm}

\section*{Código del Componente StatWrapper}

\begin{ifalgorithm}[H]
    \begin{lstlisting}
    "use client";

    import { Dialog, DialogContent, DialogHeader, DialogTitle } from "@/components/ui/dialog";
    import dynamic from "next/dynamic";

    const statComponents = {
      "hall-of-fame": dynamic(() => import("@/components/stats/HallOfFame")),
      "estaciones-musicales": dynamic(() => import("@/components/stats/EstacionesMusicales")),
      "huella-del-dia": dynamic(() => import("@/components/stats/HuellaDelDia")),
      "la-bitacora": dynamic(() => import("@/components/stats/LaBitacora")),
      "tus-decadas": dynamic(() => import("@/components/stats/TusDecadas")),
      "indice-de-interferencia": dynamic(() => import("@/components/stats/IndiceDeInterferencia")),
    };

    interface StatWrapperProps {
      activeStat: string | null;
      isOpen: boolean;
      onClose: () => void;
    }

    export default function StatWrapper({ activeStat, isOpen, onClose }: StatWrapperProps) {
      const getStatComponent = (statId: string | null) => {
        if (!statId || !(statId in statComponents)) {
          return (
            <div className='text-sm text-spotify-gray-100'>
              Selecciona una estadistica valida para ver los detalles.
            </div>
          );
        }
        const DynamicComponent = statComponents[statId as keyof typeof statComponents];
        return <DynamicComponent />;
      };

      return (
        <Dialog open={isOpen} onOpenChange={(open) => !open && onClose()}>
          <DialogContent
            className={`${
              activeStat === "tus-decadas"
                ? "w-[90vw] max-w-7xl"
                : "w-[95vw] max-w-4xl"
            } min-h-[60vh] max-h-[90vh] overflow-y-auto p-8 flex flex-col`}
          >
            <DialogHeader>
              <DialogTitle className='text-2xl font-bold text-spotify-green'>
                {activeStat
                  ? activeStat
                      .split("-")
                      .map((word) => word.charAt(0).toUpperCase() + word.slice(1))
                      .join(" ")
                  : "Estadistica"}
              </DialogTitle>
            </DialogHeader>
            <div className='flex-1 flex items-center justify-center py-4'>
              <div className='w-full mx-auto'>{getStatComponent(activeStat)}</div>
            </div>
            <div className='mt-auto text-right'>
              <button
                onClick={onClose}
                className='px-4 py-2 bg-spotify-gray-100 text-black font-semibold rounded-lg hover:bg-spotify-green/90'
              >
                Cerrar
              </button>
            </div>
          </DialogContent>
        </Dialog>
      );
    }
    \end{lstlisting}
    \caption{Definición del componente \texttt{StatWrapper}, encargado de gestionar la visualización dinámica de estadísticas en un modal.}
    \label{alg:stat_wrapper}
\end{ifalgorithm}

\section*{Código Completo del Middleware}

\begin{ifalgorithm}[H]
    \begin{lstlisting}
    import { NextResponse } from "next/server";
    import type { NextRequest } from "next/server";
    import { renovarAccessToken } from "@/lib/spotify";

    export async function middleware(req: NextRequest) {
      const access_token = req.cookies.get("access_token");
      const refresh_token = req.cookies.get("refresh_token");

      // Caso 1: Si no hay ni access_token ni refresh_token, redirir al login
      if (!access_token && !refresh_token) {
        console.log("No hay tokens, redirigiendo al login...");
        return NextResponse.redirect(new URL("/", req.url));
      }

      // Caso 2: Si hay un refresh_token pero no un access_token, renovar el token
      if (!access_token && refresh_token) {
        console.log("No hay access_token, renovando...");
        const new_tokens = await renovarAccessToken(refresh_token.value);

        if (!new_tokens) {
          return NextResponse.redirect(new URL("/", req.url));
        }

        const res = NextResponse.next();

        res.cookies.set("access_token", new_tokens.access_token, {
          httpOnly: true,
          secure: process.env.NODE_ENV === "production",
          path: "/",
          maxAge: new_tokens.expires_in, // 1 hora
        });

        if (new_tokens.refresh_token) {
          res.cookies.set("refresh_token", new_tokens.refresh_token, {
            httpOnly: true,
            secure: process.env.NODE_ENV === "production",
            path: "/",
            maxAge: 60 * 60 * 24, // 1 dia
          });
        }

        return res;
      }

      // Caso 3: Si hay un access_token, dejar que la peticion continue
      return NextResponse.next();
    }

    // Rutas protegidas
    export const config = {
      matcher: ["/home/:path*", "/stats/:path*"],
    };
    \end{lstlisting}
    \caption{Middleware global en Next.js para la gestión y renovación de tokens de autenticación.}
    \label{alg:middleware}
\end{ifalgorithm}

\begin{ifalgorithm}[H]
    \begin{lstlisting}
    export async function renovarAccessToken(refresh_token: string) {
      const clientId = process.env.SPOTIFY_CLIENT_ID!;
      const clientSecret = process.env.SPOTIFY_CLIENT_SECRET!;
      const auth_header = Buffer.from(`${clientId}:${clientSecret}`).toString("base64");

      try {
        const response = await fetch("https://accounts.spotify.com/api/token", {
          method: "POST",
          headers: {
            "Content-Type": "application/x-www-form-urlencoded",
            Authorization: `Basic ${auth_header}`,
          },
          body: new URLSearchParams({
            grant_type: "refresh_token",
            refresh_token: refresh_token,
          }),
        });

        if (!response.ok) {
          console.error("Error al renovar el token:", await response.text());
          return null;
        }

        const data = await response.json();
        console.log("Access token renovado con exito");
        console.log("Nuevos tokens:\n", data);

        return {
          ...data,
          refresh_token: data.refresh_token || refresh_token,
        };
      } catch (error) {
        console.error("Error durante la renovacion del access token:", error);
        return null;
      }
    }
    \end{lstlisting}
    \caption{Función para la renovación del \texttt{access\_token} utilizando el \texttt{refresh\_token} en la API de Spotify.}
    \label{alg:renovar_access_token}
\end{ifalgorithm}

\section*{Función de Optimización de Imagen en Hall Of Fame}

\begin{ifalgorithm}[H]
    \begin{lstlisting}
    const optimizeImage = async (canvas: HTMLCanvasElement, maxSize: number): Promise<string> => {
      let quality = 0.9;
      let imageBase64: string;
      let size: number;

      do {
        imageBase64 = canvas.toDataURL("image/jpeg", quality).split(",")[1];
        size = Math.round((imageBase64.length * 0.75) / 1024);
        console.log(`Trying quality: ${quality.toFixed(2)}, size: ${size}KB`);
        quality -= 0.1;
      } while (size > maxSize && quality > 0.1);

      if (size > maxSize) {
        // If still too large, reduce dimensions
        const scaledCanvas = document.createElement("canvas");
        const ctx = scaledCanvas.getContext("2d");
        const scale = Math.sqrt(maxSize / size);

        scaledCanvas.width = canvas.width * scale;
        scaledCanvas.height = canvas.height * scale;

        if (ctx) {
          ctx.drawImage(canvas, 0, 0, scaledCanvas.width, scaledCanvas.height);
          imageBase64 = scaledCanvas.toDataURL("image/jpeg", 0.7).split(",")[1];
          size = Math.round((imageBase64.length * 0.75) / 1024);
          console.log(`Scaled down image. Final size: ${size}KB`);
        }
      }

      return imageBase64;
    };
    \end{lstlisting}
    \caption{Función \texttt{optimizeImage()} de optimización de imágenes con ajuste de calidad y escala.}
    \label{alg:optimize_image}
\end{ifalgorithm}

\section*{Función de Creación de SVGs en Estaciones Musicales}

\begin{ifalgorithm}[H]
    \begin{lstlisting}
    const calculatePath = (startAngle: number, endAngle: number, isHovered: boolean) => {
        const r = isHovered ? expandedRadius : radius;

        const x1 = r * Math.cos((startAngle * Math.PI) / 180);
        const y1 = r * Math.sin((startAngle * Math.PI) / 180);
        const x2 = r * Math.cos((endAngle * Math.PI) / 180);
        const y2 = r * Math.sin((endAngle * Math.PI) / 180);

        const largeArcFlag = endAngle - startAngle > 180 ? 1 : 0;

        return `
          M ${x1} ${y1}
          A ${r} ${r} 0 ${largeArcFlag} 1 ${x2} ${y2}
          L ${innerRadius * Math.cos((endAngle * Math.PI) / 180)} ${innerRadius * Math.sin((endAngle * Math.PI) / 180)}
          A ${innerRadius} ${innerRadius} 0 ${largeArcFlag} 0 ${
            innerRadius * Math.cos((startAngle * Math.PI) / 180)
          } ${innerRadius * Math.sin((startAngle * Math.PI) / 180)}
          Z
        `;
    };
    \end{lstlisting}
    \caption{Cálculo de la trayectoria de los segmentos en el gráfico de \textit{Estaciones Musicales}.}
    \label{alg:calculate_path}
\end{ifalgorithm}

\section*{Funciones de Movimiento y Zoom de Tus Décadas}

\begin{ifalgorithm}[H]
    \begin{lstlisting}
    <Stage
        width={stageWidth}
        height={600}
        draggable
        scaleX={scale}
        scaleY={scale}
        x={position.x}
        y={position.y}
        onWheel={handleWheel}
        onMouseDown={() => setCursorStyle("grabbing")} // Cambia a "grabbing"
        onMouseUp={() => setCursorStyle("grab")} // Vuelve a "grab"
        onMouseLeave={() => setCursorStyle("grab")} // Asegura que el cursor vuelva a "grab" al salir
        style={{
          cursor: cursorStyle, // Usa el estado del cursor
        }}
    >
    \end{lstlisting}
    \caption{Configuración del componente \texttt{Stage} en \textit{Tus Décadas}, permitiendo navegación y zoom dinámico.}
    \label{alg:stage_tus_decadas}
\end{ifalgorithm}

\begin{ifalgorithm}[H]
    \begin{lstlisting}
    const handleWheel = (e: any) => {
        e.evt.preventDefault();

        const scaleBy = 1.2;
        const stage = e.target.getStage();
        if (!stage) return;

        const oldScale = stage.scaleX();
        const mousePointTo = {
          x: stage.getPointerPosition()!.x / oldScale - stage.x() / oldScale,
          y: stage.getPointerPosition()!.y / oldScale - stage.y() / oldScale,
        };

        const newScale = e.evt.deltaY > 0 ? oldScale / scaleBy : oldScale * scaleBy;
        setScale(newScale);

        const newPos = {
          x: -(mousePointTo.x - stage.getPointerPosition()!.x / newScale) * newScale,
          y: -(mousePointTo.y - stage.getPointerPosition()!.y / newScale) * newScale,
        };

        setPosition(newPos);
    };
    \end{lstlisting}
    \caption{Manejo del evento de desplazamiento con la rueda del ratón en \textit{Tus Décadas} para aplicar zoom dinámico.}
    \label{alg:handle_wheel_tus_decadas}
\end{ifalgorithm}

\section*{Funciones de Generación de Ondas de Índice De Interferencia}

\begin{ifalgorithm}[H]
    \begin{lstlisting}
    const generateWaveData = (frequency: number, amplitude: number = 0.5, phase: number = 0) => {
        return Array.from({ length: 200 }, (_, i) => ({
          x: i,
          y: amplitude * Math.sin(((frequency / 25) * i * Math.PI) / 24 + phase),
        }));
    };
    \end{lstlisting}
    \caption{Generación de datos de onda sinusoidal en \textit{Índice de Interferencia} a partir de la frecuencia de escucha del usuario.}
    \label{alg:generate_wave_data}
\end{ifalgorithm}

\begin{ifalgorithm}[H]
    \begin{lstlisting}
    const drawWave = (data: { x: number; y: number }[], color: string, delay: number = 0, waveType: WaveType) => {
        const path = svg
          .append("path")
          .datum(data)
          .attr("fill", "none")
          .attr("stroke", color)
          .attr("stroke-width", 2)
          .attr("d", lineGenerator)
          .attr("opacity", 0)
          .attr("class", `wave-${waveType}`);

        const totalLength = path.node()?.getTotalLength() || 0;

        path
          .attr("stroke-dasharray", totalLength + " " + totalLength)
          .attr("stroke-dashoffset", totalLength)
          .transition()
          .duration(1000)
          .delay(delay)
          .attr("opacity", 1)
          .transition()
          .duration(1500)
          .attr("stroke-dashoffset", 0);

        return path;
    };
    \end{lstlisting}
    \caption{Función \texttt{drawWave()} para dibujar ondas sinusoidales animadas en \textit{Índice de Interferencia} utilizando D3.js.}
    \label{alg:draw_wave}
\end{ifalgorithm}


